\section{Grundlegendes}
\begin{frame}{Warnhinweise}
  \begin{itemize}
    \item 100\% Sicherheit gibt es nicht
    \item Absichern heißt, Angriffe \emph{teurer} zu machen
    \begin{itemize}
    \item Die Kosten für den Angriff\\ müssen den Wert der Daten übersteigen
    \item Ein Angriff darf sich nicht mehr \emph{lohnen}
    \item Problem: Wert wird oft unterschätzt
    \end{itemize}
    \item Was wir hier zeigen, ist ein Anfang
    \begin{itemize}
      \item Hilft dagegen, als ,,Beifang`` zu enden
      \item Gegen gezielte Angriffe -- auch durch Verwechslung -- benötigt es deutlich mehr
    \end{itemize}
%    \item Irren ist menschlich -- auch was die Inhalte der folgenden Folien betrifft :-)
  \end{itemize}
\end{frame}

\begin{frame}{Leitfragen}
  \begin{itemize}
    \item \emph{Was} soll sichergestellt werden?
      \begin{itemize}
        \item Eigene Anonymität
        \item Echtheit des Gegenübers (Authentizität)
        \item Unverfälschtheit der Nachricht (Integrität)
        \item Geheimhaltung der Nachricht (Vertraulichkeit)
        \item \ldots
      \end{itemize}
    \item \emph{Wem} vertraut Ihr?
  \end{itemize}
\end{frame}

\begin{frame}{Vertrauen}
  \begin{block}{Woher weiß man, wem man vertrauen kann?}
  \begin{itemize}
    \item Kurze Antwort: weiß man \emph{nicht}
    \item Lange Antwort
    \begin{itemize}
      \item es gibt Fragen, die man stellen kann\ldots
      \item \ldots\ und es gibt das Bauchgefühl
    \end{itemize}
  \end{itemize}
  \end{block}
\end{frame}

\begin{frame}{Welche Fragen kann man stellen?}
  Beispiel: \emph{Wo} sind meine Daten?
  \begin{itemize}
    \item Auf einem Blatt Papier zuhause in meiner Schublade.
    \item Auf meinem Computer:
    \begin{itemize}
      \item Wie gut ist die Software \emph{überprüfbar},\\ die meine Daten verwaltet?
      \begin{itemize}
        \item Open Source (in menschenlesbarer Form öffentlich):\\gut überprüfbar
        \item Closed Source (menschenlesbare Form geheim):\\quasi nicht überprüfbar
      \end{itemize}
    \end{itemize}
    \item In der Cloud
      %TODO: FSF there is no cloud
      \begin{itemize}
        \item \emph{Wer} betreibt einen Dienst?
        \item Womit \emph{verdient} der Betreiber sein \emph{Geld}?
        \item Wem könnten die Daten \emph{nutzen} oder \emph{schaden}?
        \item Was \emph{lernt} der Betreiber über mich?
      \end{itemize}
  \end{itemize}
\end{frame}

\begin{frame}{Meta- und Nutzdaten}
  \begin{itemize}
    \item Meta-/Verbindungsdaten (``Briefumschlag'')
    \begin{itemize}
      \item Absender, Empfänger, Betreff einer E-Mail
      \item Besuch und Aufenthaltsdauer auf einer Webseite
      \item Wer, wann, wie lange mit wem telefoniert
      \item Aufenthaltsort von Mobiltelefonen: Bewegungsprofil!
    \end{itemize}
    \item Nutz-/Inhaltsdaten (``Brief'')
    \begin{itemize}
      \item E-Mail-Text und -Anhänge
      \item Webseiten-Inhalte
      \item Gesprochene Sprache beim Telefonieren
      \item SMS-Inhalt
    \end{itemize}
  \end{itemize}

Metadaten zu verschlüsseln ist nicht möglich,\\ sie zu verschleiern schwierig.
\end{frame}

\begin{frame}{Die Cryptoparty}
  \begin{itemize}
    \item Weltweite Bewegung von technisch interessierten
    \item Ziel: Datensicherheit für jedermann
    \item Themen sind z.B.
    \begin{itemize}
      \item Kommunikation: E-Mail, Anrufe, Chat
      \item Datenspeicherung und -weitergabe
      \item Veröffentlichen von Informationen
      \item Passwörter
    \end{itemize}
    \item Aus Zeitgründen beschränken wir uns heute auf 
    \begin{itemize}
      \item Passwort-Management
      \item Anonyme(re)s Web-Surfen
      \item E-Mail-Verschlüsselung und -Signatur
      \item Messenger
      \item \ldots und mehr auf Anfrage, wenn noch Zeit ist
    \end{itemize}
  \end{itemize}
\end{frame}



\endinput
