\section{Messenger}
\begin{frame}{Motivation}
\begin{itemize}
\item Komfortabel, auf Smartphone einfach nutzbar
\item Wird im privaten Umfeld meist häufiger eingesetzt\\ als E-Mail
\end{itemize}

\pause
\begin{block}{Bestandsaufnahme}
Wer benutzt
\begin{itemize}
\item<+-> WhatsApp
\item<+-> Telegram
\item<+-> Threema
\item<+-> Signal
\item<+-> Jabber
\item<+-> Matrix
\end{itemize}
\end{block}
\end{frame}

\begin{frame}{Überblick}
  \begin{itemize}
    \item Geschlossene Systeme: WhatsApp \& Co
      \begin{itemize}
        \item App-Entwickler\\
          ist Dienstanbieter\\
          und Herr über die ``technische Sprache'' (das Protokoll)
        \item Auswahl eines Dienstes\\bestimmt erreichbaren Personenkreis
        \item meist Closed Source
      \end{itemize}
    \item Offene Systeme (z.B. Jabber/XMPP, Matrix)
      \begin{itemize}
        \item App-Entwickler, Dienstanbieter und Protokoll-Standardisierer sind unterschiedliche Personen
        \item App und Anbieter frei wählbar
        \item meist Open Source
      \end{itemize}
  \end{itemize}
\end{frame}

\setbeamersize{description width=1em}

\begin{frame}{WhatsApp}
\begin{blex}{Vor- und Nachteile}
\item[+] Angeblich sehr gute Crypto (von Signal eingekauft)
\item[+] Verwendbar ohne Google Play
\item[-] Closed Source
\item[-] Anbieter erhält eine vollständiges Telefonbuchs\\
  (nicht nur WhatsApp-Kontakte)
\item[-] Metadatenanalyse, -weitergabe an Facebook
\end{blex}
\end{frame}

\begin{frame}{Telegram}
\begin{blex}{Vor- und Nachteile}
\item[+] \glqq Sichere Chats\grqq\ erfüllen Kriterien der EFF
\item[o] Überprüfung der Schlüssel über Muster
\item[-] Chats nicht standardmäßig verschlüsselt
\item[-] Anbieter erhält vollständiges Telefonbuch\\ (nicht nur Telegram-Kontakte)
\item[-] \glqq Normale Chats\grqq\ werden im Klartext beim Anbieter\\ gespeichert und ausgiebig analysiert
\end{blex}
\end{frame}

\begin{frame}{Threema}
\begin{blex}{Vor- und Nachteile}
\item[+] Chats erfüllen angeblich Kriterien der EFF\\ (nicht überprüfbar weil closed Source)
\item[+] Synchronisation des Telefonbuchs optional
\item[+] Überprüfung der Schlüssel über QR-Code
\item[o] Kostenpflichtig
\item[-] Closed Source
\end{blex}
\end{frame}

\begin{frame}{Signal}
\begin{blex}{Vor- und Nachteile}
\item \glqq Snowden approved\grqq
\item[+] Chats standardmäßig sehr gut verschlüsselt
\item[+] Keine Metadatenaufzeichnung
\item[+] Überprüfung der Schlüssel über QR-Code
\item[-] Ohne Google Play Services eingeschränkte Funktion 
\item[-] Anbieter erhält komplettes Telefonbuch\\ (nicht nur Signal Kontakte)
\end{blex}
Genauere Sicherheitsanalyse: {\small\url{https://blog.fefe.de/?ts=a9226eda}}
\end{frame}

\begin{frame}{Zusammenfassung}
\begin{blit}{Detailinfos zu vielen Messengern}
\item EFF Secure Messaging Scorecard\\ {\small\url{https://www.eff.org/secure-messaging-scorecard}}
\end{blit}

\begin{itemize}
\item Ab Android 6: Zugriff auf Telefonbuch verbieten
\begin{itemize}
\item Dann aber nur noch Nummern statt Namen
\end{itemize}
\end{itemize}
\end{frame}
