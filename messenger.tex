\section{Messenger}
\begin{frame}{Motivation}
  \begin{itemize}
    \item Komfortabel, auf Smartphone einfach nutzbar
    \item Wird im privaten Umfeld meist häufiger eingesetzt\\ als E-Mail
  \end{itemize}

  \pause
  \begin{block}{Bestandsaufnahme}
    Wer benutzt
    \begin{itemize}
      \item<+-> WhatsApp
      \item<+-> Telegram
      \item<+-> Threema
      \item<+-> Signal
      \item<+-> Jabber
      \item<+-> Matrix
    \end{itemize}
  \end{block}
\end{frame}

\begin{frame}{Überblick}
  \begin{itemize}
    \item Geschlossene Systeme: WhatsApp \& Co
      \begin{itemize}
        \item \emph{Eine Firma} kontrolliert\\ App, Protokoll und ist Dienstanbieter
        \item Auswahl eines Dienstes\\bestimmt erreichbaren Personenkreis
        \item Meist closed-source
      \end{itemize}
    \item Offene Systeme (z.B. Jabber/XMPP, Matrix)
      \begin{itemize}
        \item App, Dienstanbieter und Protokoll kommen\\ von \emph{unterschiedlichen} Firmen bzw. Personen
        \item Alle Nutzer des Protokolls können erreicht werden, egal welche App sie nutzen\\ und bei welchem Dienstanbieter sie sind
        \item Meist open-source
      \end{itemize}
  \end{itemize}
\end{frame}

\setbeamersize{description width=1em}

\begin{frame}{WhatsApp}
  \begin{blex}{Vor- und Nachteile}
    \item[+] Angeblich sehr gute Crypto (von Signal eingekauft)
    \item[+] Verwendbar ohne Google Play Services
    \item[-] Closed-source
    \item[-] Anbieter erhält vollständiges Telefonbuch\\
             (nicht nur WhatsApp-Kontakte)
    \item[-] Metadatenanalyse, -weitergabe an Facebook
  \end{blex}
  \begin{block}{Tipps}
    \begin{itemize}
      \item Ab Android 6 und bei iOS kann man für WhatsApp\\den Zugang zum Telefonbuch sperren
      \item Android 8 Dual Messenger erlaubt\\selektive Freigabe von Kontakten\\
            \scriptsize nicht getestet, nur Samsung(?)
    \end{itemize}
  \end{block}
\end{frame}

\begin{frame}{Telegram}
  \begin{blex}{Vor- und Nachteile}
    \item[+] \glqq Sichere Chats\grqq\ bieten ausreichende Sicherheit
    \item[-] Chats nicht standardmäßig verschlüsselt
    \item[-] Anbieter erhält vollständiges Telefonbuch\\ (nicht nur Telegram-Kontakte)
    \item[-] \glqq Standard-Chats\grqq\ werden im Klartext\\beim Anbieter gespeichert
  \end{blex}
\end{frame}

\begin{frame}{Threema}
  \begin{blex}{Vor- und Nachteile}
    \item[+] Chats hinreichend gut verschlüsselt\\(aber nicht gut überprüfbar, da nicht quelloffen)
    \item[+] Synchronisation des Telefonbuchs optional
    \item[+] Überprüfung der Schlüssel über QR-Code
    \item[o] kostenpflichtig
    \item[-] nicht quelloffen
  \end{blex}
\end{frame}

\begin{frame}{Signal}
  \begin{blex}{Vor- und Nachteile}
    \item[+] Chats standardmäßig sehr gut verschlüsselt
    \item[+] Überprüfung der Schlüssel über QR-Code
    \item[-] Anbieter erhält komplettes Telefonbuch\\ (nicht nur Signal-Kontakte)
  \end{blex}
\end{frame}

\begin{frame}{Jabber/XMPP}
  \begin{blex}{Vor- und Nachteile}
    \item[+] Offenes System: Anbieter und App frei wählbar
    \item[+] Verschlüsselung möglich (OTR oder OMEMO)
    \item[+] keine Telefonbuch-Synchronisation vorgesehen
    \item[-] Crypto nicht ganz so nutzerfreundlich\\wie bei kommerziellen Anbietern\\(aber dafür sind wir ja alle hier :-)
  \end{blex}
  \begin{itemize}
    \item Apps: Conversations, pidgin, gajim, \ldots
    \item Anbieter: Unis, Hackerspaces, CCC, jabber.org, \ldots
  \end{itemize}
\end{frame}

\begin{frame}{Matrix}
  \begin{blex}{Vor- und Nachteile}
    \item[+] Offenes System: Anbieter frei wählbar
    \item[+] Verschlüsselung möglich
    \item[+] Telefonbuch-Synchronisation nicht zwingend
    \item[+] Anbindung an Drittsysteme möglich (IRC, Skype, \ldots)
    \item[-] Hintergrund/Historie der Entwickler umstritten
  \end{blex}
\end{frame}

\begin{frame}{Zusammenfassung}
  \begin{itemize}
    \item Viele beliebte Messenger\\ sind geschlossene Systeme
    \item Darauf achten, womit der Anbieter sein Geld verdient
    \item Wer auf bestimmte Messenger nicht verzichten kann: Zugriff auf Kontakte verbieten!
  \end{itemize}
\end{frame}
