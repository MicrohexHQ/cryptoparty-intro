\section{Web-Surfen}
\begin{frame}{Tracking}
  \begin{block}{Ausgangslage}
    Ein Großteil aller Websites nutzen Drittanbieter für
    \begin{itemize}
      \item Werbung
      \item ``Social Media''-Präsenz
      \item statistische Analysen
      \item \ldots
    \end{itemize}
  \end{block}

    \pause
    Diese Drittanbieter
    \begin{itemize}
      \item bekommen jeden Besuch einer solchen Webseite mit
      \pause
      \item können Besucher seitenübergreifend identifizieren
      \pause
      \item sind oft Dienstleister für sehr viele Webseiten gleichzeitig
    \end{itemize}
    \pause
    Ja und?
    \pause
    \begin{itemize}
      \item wenige Anbieter lernen sehr viel über jeden Nutzer
      \pause
      \item \ldots und verteilen manchmal (unfreiwillig) Schadsoftware
    \end{itemize}
  \end{frame}
  \begin{frame}{Tracking}
  \framesubtitle{Technische Umsetzung}
  \begin{itemize}
    \item IP-Adresse
    \item Cookies und Co (HTML5 Persistent Local Storage, Flashcookies, \ldots)\\
      \scriptsize Visualisierung: \url{http://datenblumen.wired.de/} \normalsize
    \item Browser-Fingerabdruck\\
      \scriptsize Visualisierung: \url{https://panopticlick.eff.org/} \normalsize
  \end{itemize}
\end{frame}

\begin{frame}{Schutzmaßnahmen -- Level 1}
\framesubtitle{Nur Einstellungen ändern}
  \begin{itemize}
    \item Standardsuchmaschine\\ auf datenschutzfreundliche Anbieter ändern, z.B.
    \begin{itemize}
      \item DuckDuckGo
      \item Startpage
    \end{itemize}
    \item Cookies verbieten oder nur selektiv erlauben
    \item Plugins auf ,,Click-to-use`` stellen
    \item Verlauf beim Beenden löschen
  \end{itemize}

  %Eventuell:
  %\begin{itemize}
  %  \item Blockierung von ,,bösen`` Webseiten abschalten
  %  \item Statusberichte des Browsers abschalten
  %\end{itemize}
  % https://support.mozilla.org/en-US/kb/how-does-phishing-and-malware-protection-work#w_what-information-is-sent-to-mozilla-or-its-partners-when-phishing-and-malware-protection-are-enabled
\end{frame}

\begin{frame}{Schutzmaßnahmen -- Level 2}
\framesubtitle{Plug-Ins installieren}
  \begin{itemize}
    \item Adblocker
    \item Tracking-Blocker
    \item Cross-Domain-Request-Blocker
    \item Referer-Manager
    \item \ldots
  \end{itemize}
  \pause
  \textbf{Details später in Kleingruppen}
\end{frame}

\begin{frame}{Schutzmaßnahmen -- Level 3}
\framesubtitle{Neue Programme installieren oder benutzen}
  \begin{itemize}
    \item \href{https://www.torproject.org}{Tor Browser Bundle (Freie Software)}
    \begin{itemize}
      \item Anonymisierung des Webverkehrs\\durch ,,intelligente Umwege''\\
        \scriptsize \url{https://www.torproject.org/about/overview} \normalsize
      \item Fingerabdruck bei allen Tor Browsern identisch
      \item Gewählte Plugins vorinstalliert
      \item Automatische Updates
      \item Hohe Sicherheit, aber prinzipbedingt langsamer
    \end{itemize}
    \pause
    \item \href{https://tails.boum.org}{Tails (Freie Software)}
    \begin{itemize}
      \item Abgesichertes Betriebssystem inkl. Tor
      \item Live System = kann direkt von CD gebootet werden\\ hinterlässt keinerlei Spuren am PC
      % As of 2016-Apr-14, Windows Camouflage has been removed in Tails 2 and newer
      %   https://tails.boum.org/news/windows_camouflage_jessie/index.en.html
      %   https://tails.boum.org/blueprint/update_camouflage_for_jessie/
      %\item Sieht auf Wunsch nach Windows aus
      \item Leitet gesamten Verkehr über Tor
    \end{itemize}
  \end{itemize}
\end{frame}

\endinput
